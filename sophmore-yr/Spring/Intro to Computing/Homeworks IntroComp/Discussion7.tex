\documentclass{article}
\usepackage{listings}
\usepackage{color}

\definecolor{dkgreen}{rgb}{0,0.6,0}
\definecolor{gray}{rgb}{0.5,0.5,0.5}
\definecolor{mauve}{rgb}{0.58,0,0.82}

\lstset{frame=tb,
  language=Java,
  aboveskip=3mm,
  belowskip=3mm,
  showstringspaces=false,
  columns=flexible,
  basicstyle={\small\ttfamily},
  numbers=none,
  numberstyle=\tiny\color{gray},
  keywordstyle=\color{blue},
  commentstyle=\color{dkgreen},
  stringstyle=\color{mauve},
  breaklines=true,
  breakatwhitespace=true,
  tabsize=3
}


\begin{document}

\title{Java Methods Implementation}
\author{Aidan Chin, Rayan Azad, Anhad Singh, Seth Oswalt}
\maketitle

\section*{Writing Exercise}

Write (in real Java) the following methods to be added to this class:
\begin{lstlisting}[language=Java]
    public class BooleanExpression {
        public static final int AND = 1; // AND operator
        public static final int OR = 2; // OR operator
        public static final int NOT = 3; // NOT operator
        
        boolean isLeaf; // true if this is a one-node tree
        boolean leafValue; // value of tree if it has just one node
        int operator; // value must be AND, OR, or NOT
        BooleanExpression left, right; // subexpressions if !isLeaf
        
        public BooleanExpression (boolean arg) {
            // create one-node tree with given value
            isLeaf = true;
            leafValue = arg;
            left = right = null;}

        public BooleanExpression (int op, BooleanExpression leftArg, 
                                  BooleanExpression rightArg) {
            // create tree with given operator, left argument, right argument
            isLeaf = false;
            operator = op;
            left = leftArg;
            right = rightArg;}

        public boolean getIsLeaf () {return isLeaf;}
        public boolean getLeafValue ( ) {return leafValue;}
        public int getOperator () {return operator;}
        public BooleanExpression getLeft () {return left;}
        public BooleanExpression getRight () {return right;}}
\end{lstlisting}

\begin{enumerate}
    \item A method \texttt{size} that returns an \texttt{int} giving the number of nodes in the calling expression’s tree. (We’ll mostly do this on the blackboard.)
    \item A method \texttt{leaves} that returns an \texttt{int} giving the number of leaves in the calling expression’s tree.
    \item A method \texttt{depth} that returns an \texttt{int} giving the depth of the tree, which is the number of nodes in the longest directed path from the root node to any leaf.
    \item A method \texttt{eval} to return the boolean value of the calling expression.
    \item (if time) A method \texttt{toString} that returns a \texttt{String} representing the expression, in an infix format like “NOT ((true OR false) AND (false OR true))”.
\end{enumerate}

\section*{Response Area}

\begin{lstlisting}[language=Java]
class JavaMethods:
    public int size():
        {
            if (getIsLeaf()) return 1;
            if (getOperator() == NOT) return 1 + getLeft.size();
            return 1 + getLeft.size() + getRight.size();
        }
        
    
    public int leaves():
        {
            if(getIsLeaf()) return 1;
            return getLeft.leaves() + getRight.leaves();
        }
    
    public int depth(self):
        {
            if(getIsLeaf()) return 0;
            int leftDepth = getLeft.depth();
            int rightDepth = getRight.depth();
            return max(leftDepth, rightDepth);
        }
    
    public boolean eval(self):
        {

        }
    
    public str toString(self):
        # Write your Java code for toString method here
        pass
\end{lstlisting}

\end{document}
